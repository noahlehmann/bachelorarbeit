% Einleitung
\chapter{Einleitung}
\label{sec:einleitung}

% Text

Die Digitalisierung ist im zweiten Jahrzehnt des 21. Jahrhunderts kaum noch aufzuhalten. Was früher aufwändig manuell gesammelt und dargestellt wurde, hält heute Einzug in den Smartphones der Menschen. So wird zur Navigation keine Karte mehr benötigt, lediglich ein \ac{GPS}-fähiges Handy. Zum Einkaufen muss man sich keinen Merkzettel mehr schreiben, sondern dem Smart-Assistenten diktieren, was man braucht. Dieser speichert alles automatisiert auf dem Smartphone, griffbereit sobald man im Laden angekommen ist. Die Nachrichten liest man nicht mehr in der Zeitung, sondern im News-Feed. Das Wetter muss man nicht mehr in den Nachrichten aufschnappen, viel einfacher ist die minutengenaue Aktualisierung der Daten auf dem digitalen Begleiter.
\\
\linebreak
Dieser Trend hat sich in den vergangenen Jahren auch in den Lehrinstituten durchgesetzt. Somit wurden im Zuge der Digitalisierung die Mechanismen neu überdacht und durch modernere Alternativen ersetzt. Was früher mit Kreide an die Tafeln geschrieben wurde, wird heute durch den Beamer an die Wand projiziert. Die Zettel, die einst vielfach gedruckt und verteilt werden mussten, werden durch Lernplattformen wie Moodle und Chamilo ersetzt. Dort können die Dokumente einfach hochgeladen und verteilt werden. Die Informationen, die man damals noch im Sekretariat erfragen musste, werden heute auf den jeweiligen Internet Seiten dargestellt. Und so steigen die Schulen und Hochschulen auch langsam auf die Digitalisierung im Bereich des Stundenplans um. Man hat sich an die Bedürfnisse der Studierenden angepasst. Im Laufe der letzten beiden Jahrzehnte wurden für weite Teile der Organisation der Institutionen digitale Helfer eingeführt. Diese reichen von Datenbanken zum Speichern von Zeitplänen und anderem bis zu \acp{App} zum erleichterten Anzeigen von wichtigen Informationen wie Stundenplan Daten. 

\section{Beweggründe}

So wurde auch an der Hochschule Hof - im Zuge dieser Bewegung - die Bereitstellung der Stundenplaninformationen für die Studierenden weitgehend erleichtert. Bereits der Zugang des Stundenplans über die Website war ein großer Fortschritt, doch wurde auf Antrieb der Fakultät Informatik sogar die Entwicklung nativer Anwendungen für das Smartphone angestoßen. Diese ermöglichten es den Studierenden die nötigen Zeiten und Veranstaltungsorte ihrer Vorlesungen jederzeit über das Smartphone  abzurufen. Des weiteren wurden die Funktionalitäten dieser \acp{App} über die Jahre so erweitert, dass sie bis zuletzt zusätzliche Funktionen wie Speiseplan Informationen, Raumsuche und Kalendereinbindung anbieten konnten. Durch diese Entwicklungen konnten bis zum Jahr 2019 mehr als 90\% der Studierenden der Hochschule Hof dazu bewegt werden, eine der angebotenen Smartphone Anwendungen zu nutzen. Lediglich 7\% nutzen immer noch die Website zum Suchen ihrer Veranstaltungen\footnote{Siehe Kapitel \ref{sec:umfrage}}.
\\
\linebreak
Allerdings gibt es bei den Nutzungsstatistiken der hochschuleigenen \acp{App} noch große Unterschiede zwischen den einzelnen Nutzergruppen, speziell den internationalen Studierenden und denen, die dauerhaft an der Hochschule eingeschrieben sind. So nutzen lediglich 75\% der internationalen Studierenden eine der Anwendungen der Hochschule, der Rest bedient sich bei den Informationen bei der Webseite. Zudem berichten auch nur 30\% der Internationalen Studierenden, dass sie ohne größeren Aufwand einen vollständigen Stundenplan einrichten können. Des weiteren wird angemerkt, dass viele Informationen in den Hochschul-\acp{App} nur in deutscher Sprache aufgeführt werden, was es anderssprachigen Studierenden deutlich schwerer macht, die Inhalte korrekt zu interpretieren. Doch auch die deutschsprachigen Studierenden haben im Rahmen der Umfrage knapp fünfzig Verbesserungsvorschläge erbracht und auch klar gemacht, dass eine verbesserte Version der aktuellen \acp{App} deutlich öfter genutzt werden würde.
\\
\linebreak
Zusätzlich zum Funktionsumfang der bereits vorhandenen Anwendungen sind die Entwicklungsumstände dieser \acp{App} im aktuellen Zustand zu bemängeln. Eines der Probleme liegt darin, dass die Hochschule für drei Betriebssysteme jeweils eine eigene \ac{App} anbietet. So ergibt sich die dreifache Entwicklungsarbeit, denn es muss jeweils eine Anwendung für Android, \ac{iOS} und Windows gepflegt werden. In einigen Bereichen der Entwicklung hat der Einsatz in der vergangenen Zeit stark nachgelassen, zumal die Hauptentwickler der stark genutzten Android \ac{App} in den vergangenen Semestern die Hochschule verlassen haben. 

\section{Zielsetzung}

Aus den genannten Problemen der Hochschul-\acp{App} hat sich das Thema dieser Bachelorarbeit geformt. Diese wird sich um die Entwicklung einer Lösung für diese Schwierigkeiten kümmern. 
\\
\linebreak
Anhand von Umfragen, die an die Nutzer der Hochschul-\acp{App} gerichtet sind, soll analysiert werden, welche Aspekte einer Überarbeitung bedürfen und welche aus Vorhandenem erfüllt werden können. Des weiteren werden die aktuellen \acp{App} der Hochschule Hof kritisch betrachtet. Mithilfe des International Office und des Sprachenzentrums, sowie des Auftraggebers Prof. Dr. Heym werden dann die Anforderungen an eine neue Anwendung zur Darstellung von Stundenplan Daten, Terminänderungen, Mensa-Speiseplänen und möglichen weiteren Informationen gesammelt. Diese funktionalen Anforderungen werden dann auf ihre Machbarkeit geprüft und schlussendlich schrittweise in einen Prototyp für eine plattformübergreifende Hochschul-\ac{App} übernommen. Die genauen Funktionen dieser \ac{App} werden aufgeführt und erläutert, worauf die Vorgehensweise für die Implementierung dieser dargestellt wird. Der Fokus dabei liegt auf den angewandten Techniken und Frameworks, welche mit ihren Alternativen verglichen werden, wobei der Entscheidungsvorgang zu den jeweiligen Techniken erarbeitet wird. Daraufhin soll die Anwendung als Ganzes betrachtet werden, was eine genauere Erschließung der Architektur erfordert, auf die diese aufbaut. Abschließend soll dann betrachtet werden, welche Funktionen im Rahmen dieser Arbeit implementiert werden können und wie die \ac{App} in Zukunft noch weiter entwickelt werden kann.

\section{Zielgruppe}

Im allgemeinen behandelt diese Arbeit die Entwicklungsphase und die Vorbereitung einer neuen und verbesserten, plattformunabhängigen Hochschul-\ac{App}. In den ersten Kapiteln werden also die Rahmenbedingungen für die Entwicklung gesetzt. Es werden unter anderem die bereits vorhandenen Anwendungen der Hochschule analysiert und die Zufriedenheit der Nutzer dieser \acp{App} genauer betrachtet. Danach werden anhand der gesammelten Ergebnisse, aber auch durch gegebene Prämissen, die allgemeinen funktionalen Anforderungen aufgeführt und erläutert. Dieser Teil der Bachelorarbeit ist für eine breite Gruppe an Lesern interessant. Anhand der Umfrage zur Zufriedenheit können beispielsweise die Entwickler der vorhandenen \acp{App} ihre Schlüsse ziehen und die eigenen Projekte verbessern, durch die Sammlung der Anforderungen hingegen wird das Projekt und dessen Umfang genau definiert, was besonders für den Auftraggeber und die beteiligten Entwickler von Bedeutung ist.
\\
\linebreak
Der zweite Teil der Arbeit geht dann genauer auf die Umsetzung des Projektes ein, es wird erklärt, in welche Teile das Projekt aufgeteilt wird und warum diese Teilung vorgenommen wird. Anhand dieser Teilung wird dann das Rückgrat des Projektes, die Datenquelle entworfen. Mit diesem Teil können die Entwickler des Projektes genau einsehen, warum Entscheidungen zu den genutzten Techniken getroffen wurden und was das für das gesamte Projekt bedeutet. Für spätere Arbeiten an dem Projekt ist der zweite Teil der Arbeit auch ein guter Einstieg in die Thematik und bildet eine gute Grundlage zu dem benötigten Wissen über die genutzten Design Patterns. 
\\
\linebreak
Im letzten Teil der Arbeit wird dann eine konkrete Lösung des Problems erarbeitet. Dabei werden alle relevanten Architekturentscheidungen getroffen und erläutert, sowie die nötigen Schnittstellen definiert. Dieser Teil der Bachelorarbeit bietet die Grundlage für die Entwicklung des eigentlichen Prototypen. Entwickler in allen Stadien des Projektes können sich durch diese Kapitel einlesen und die Grundstruktur des Projektes verstehen. Genaueres zur Vorgehensweise und den benutzten Techniken findet man allerdings in der zu dieser Bachelorarbeit gehörigen Praxisarbeit, die im Kapitel \ref{sec:nebenlektuere} nochmals erwähnt wird.

\section{Vorausgesetztes Wissen}

Im allgemeinen ist diese Bachelorarbeit eine Arbeit aus dem Bereich der Informatik, dementsprechend ist ein solides Grundwissen im Bereich des Informationsmanagements, der Programmierung und der Datenspeicherung von Vorteil. Grundlegende Konzepte des Software Entwurfs sollten dem Leser ebenfalls bekannt sein. Des weiteren analysiert diese Arbeit mehrere Ideen aus dem Bereich der Webentwicklung, wobei dort eher allgemeine Konzepte vorgestellt werden. Genauere Umsetzungen werden in den darauf folgenden Kapiteln genauer beleuchtet und erläutert. Basierend auf den Webtechnologien, die angeschnitten werden, sollten dem Leser die gängigsten Datenformate und Kommunikationsprotokolle aus dem Webbereich ein Begriff sein. Wo fundiertes Wissen voraus gesetzt wird, wird ebenfalls eine kurze Zusammenfassung des benötigten Wissens geliefert.
\\
\linebreak
Trotz der technischen Aspekte dieser Arbeit kann ein großer Teil auch von Lesern verstanden werden, die kein Wissen aus der Informatik oder speziell aus den oben genannten Bereichen vorweisen können. Die ersten Kapitel befassen sich, wie bereits erwähnt, mit der Analyse der vorhandenen Anwendungen der Hochschule Hof, sowie mit der Zufriedenheit der Nutzer dieser Anwendungen. Daraufhin werden die funktionalen Anforderungen gesammelt - auch dazu wird kein fundiertes Wissen aus der Informatik benötigt. Der zweite Teil der Arbeit befasst sich dann mit der Analyse von verschiedenen Konzepten, wo ein fundiertes Grundwissen von Vorteil ist, aber nicht zwingend notwendig. Auch hier werden viele Erklärungen geliefert, um dem Leser eine gute Grundlage für das Verständnis der letzten Kapitel zu geben. In diesen werden dann genaue Konzepte und Lösungsansätze erarbeitet, die überwiegend technisch versiert sind. Spätestens dort werden, wie oben erwähnt, die fundierten Grundlagen im Bereich Informatik benötigt, um die erarbeiteten Lösungen vollständig nachvollziehen zu können.

\section{Vorgeschlagene Nebenlektüre\label{sec:nebenlektuere}}

Um ein bestmögliches Verständnis dieser Arbeit zu erlangen, ist es ratsam einige andere Quellen und Arbeiten zum Thema der Entwicklung der Hochschul Anwendung und zu anderen technischen Themen zu lesen.

Im Rahmen des Projektes der Neuentwicklung einer Hochschul-\ac{App} werden mehrere Arbeiten verfasst, welche sowohl die Anforderungen analysieren, als auch die Vorgehensweise beschreiben. Eine dieser Arbeiten ist eine weitere Bachelorarbeit, welche sich mit der Anwenderverwaltung, Autorisierung und mit der grafischen Oberfläche des Prototypen befasst\autocite[][]{andreasba}. Dort werden einige Teile der Gesamtarbeit analysiert, welche zur Vollständigkeit dieser Arbeit und um den Lesefluss zu erhalten, in dieser Bachelorarbeit nur kurz beschrieben und referenziert werden.
\\
\linebreak
Des weiteren werden zu beiden Bachelorarbeiten, sowohl zu dieser, als auch zur zweiten innerhalb dieses Projektes\autocite[][]{andreasba}, Praxisarbeiten angefertigt, welche umfangreiche Dokumentationen zum Aufbau und zur Umsetzung der Implementierung des Prototypen der Hochschul-\ac{App} beinhalten. Dort werden alle Frameworks und technischen Komponenten näher erläutert. Des weiteren wird dort die Vorgehensweise bei der Implementierung beleuchtet, welche näheren Einblick auf die Entwicklung geben soll, sodass der Umfang und die umgesetzten - beziehungsweise die ausgelassenen - Funktionen und Komponenten des Prototypen gerechtfertigt werden können\autocites[][]{dnpa}[][]{andreaspa}.
\\
\linebreak
Des weiteren finden sich zu dem Thema \textit{Modulare Web-Architektur} einige gute Bücher, die den Einstieg deutlich vereinfachen und die im Rahmen dieser Arbeit auch referenziert werden. Dazu gehört beispielsweise der Titel \textit{Grundlagen des modularen Softwareentwurfs}\autocite[][]{gmodse} von Herbert Dowalil, welcher ausführlich auf die Konzepte \textit{Service-orientierte Architektur} und \textit{Microservices} eingeht. Wer jedoch einen allgemeineren Blick auf die Konzepte der Web-Technologien erlangen möchte, die in dieser Arbeit analysiert werden, dem wird das Buch \textit{Verteilte Systeme}\autocite[][]{verteiltesys} eine bessere Einsicht verschaffen. Weitere Bücher zu diesem und zu den in dieser Arbeit relevanten Themen sind dem Literaturverzeichnis zu entnehmen.