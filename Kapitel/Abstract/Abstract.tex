\thispagestyle{empty}
\newgeometry{left=2.5cm, right=2.5cm, bottom=2.5cm, top=2.5cm}

\section*{Abstract}
Die vorliegende Bachelorarbeit zielt darauf ab, die Analyse für eine zukunftsfähige, beständige und erweiterbare Hochschulapp bereitzustellen. Um eine solide Grundlage dafür zu schaffen, werden in dieser Arbeit ebenfalls bestehende Anwendungen zur Informationsbeschaffung der Hochschule Hof analysiert und die Nutzergruppen zu deren Präferenzen  befragt. Aus den gewonnenen Ergebnissen werden dann die Anforderungen gesammelt und anhand derer die finalen Entwürfe der Anwendung gefertigt und gerechtfertigt.\\
Der erste Teil der Arbeit befasst sich lediglich mit der Analyse der vorhandenen plattformgebundenen Anwendungen der Hochschule Hof, unter anderem der Android App, der iOS App aber auch mit der klassischen Website der Hochschule Hof. Es werden die Funktionen gesammelt und analysiert. Danach werden die Nutzer der Apps, die Studierenden der Hochschule Hof, zu den bestehenden Anwendungen befragt. Anhand der gewonnenen Erkenntnisse können die gewünschten Funktionen dann gesammelt und bewertet werden.\\
Mit den Informationen aus dem ersten Teil der Arbeit werden dann die Anforderungen an die Arbeit gesammelt. Dabei wird nicht nur der Auftraggeber mit einbezogen, sondern auch andere Interessengruppen, genauer das International Office und das Sprachenzentrum der Hochschule Hof. Zu den wichtigsten Anforderungen kommt die Plattformunabhängigkeit der Anwendung und die Internationalisierung in Form von Mehrsprachigkeit im Einklang mit dem Leitbild der Hochschule Hof. Zu den Anforderungen wird außerdem ein referenzierbares Lastenheft angefertigt.\\
Um die nicht funktionalen Anforderungen wie Erweiterbarkeit und Modularität gerecht zu werden, wird im Mittelteil der Arbeit der Fokus auf einige Prinzipien und Designphilosophien der Softwarearchitektur gelegt. Hier werden Ideen wie das KISS-Prinzip, Serviceorientierte Architektur und REST  genauer betrachtet, teils auch mit Alternativen verglichen, um die passende Lösung für einen Prototypen einer neuen Hochschul-App zu finden. Durch die Identifizierung RESTful Webservices als Architekturgrundlage wird im Anschluss an die Sammlung der Paradigmen ebenfalls ein Regelwerk für das Arbeiten mit REST festgelegt.\\
Abschließend wird konkret an der Lösung der Hochschul-App gearbeitet. Hierfür wird erst die grundlegende Architektur angerissen und genauer erläutert. Lösungen wie Discovery Services und API-Gateways werden dort genauer erläutert. Abschließend wird dann der konkrete Aufbau der Microservice Architektur erklärt. Hierbei werden nicht nur die Ressourcen definiert, es wird auch auf die darunterliegende Logik und den Datenbankkonzepten eingegangen.\\
Zusammenfassend erarbeitet diese Bachelorarbeit den konkreten Prototypen der Hochschul-App in allen Phasen der Analyse. Durch die Ideen aus den vorhandenen Anwendungen und dem Feedback der Nutzer konnten die Anforderungen gesammelt werden. Aus diesen Anforderungen konnten dann die nötigen Architekturprinzipien identifiziert werden und daraus resultierend die passende Architektur gefunden werden. Aufbauend auf der Architektur konnten dann die Feinheiten der App ausgearbeitet werden.