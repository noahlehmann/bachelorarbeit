% Microservices Frameworks
\subsection{Microservices Frameworks}
\label{sec:frameworks}
\todo[inline]{Grundlagen des Modularen Softwareentwurfs Vorwort XVI (Technologie/Betrieb)}
% Text
Die Auswahl der einzelnen Technologien wie Frameworks ist so eng mit dem späteren Betrieb der Software verbunden.
Um das ultimative Ziel einer jeder Architekturdefinition nicht zu gefährden, nämlich die langlebige Wartbarkeit des Systems, sollte es dabei so gut es eben möglich ist immer gewährleistet sein, einzelne Technologien später auszutauschen.
% Microservices Frameworks Varianten
\subsubsection{Varianten}
\label{sec:frameworks_varianten}

% Text

% Microservices Frameworks Bewertung
\subsubsection{Bewertung}
\label{sec:frameworks_bewertung}

% Text

% Microservices Frameworks Entscheidung
\subsubsection{Entscheidung}
\label{sec:frameworks_entscheidung}

% Text
\todo[inline]{Grundlagen des Modularen Softwareentwurfs Technologien 1.1.4}
Es sollte tendenziell immer auf langweiligste und am besten etablierte Technologie gesetzt werden, die zur Umsetzung der jeweiligen Anforderungen ausreichend ist. Etablierte Technologien haben diverse Vorteile, weil man bsp schon die diversen Vor- und Nachteile gut einschätzen kann. Es gibt meist recht viel Personal am Arbeitsmarkt, das damit umgehen kann. Und wenn es schon länger existiert, stehen die Chancen gut, dass es auch weiterhin Support dafür geben wird. Viele "Shiny New Objects" aus dem Internet, weil neue Frameworks zur Entwicklung von Web-Guis, haben teilweise eine kürzere Haltbarkeitsdauer als die meisten Dinge aus dem nächstbesten Supermarkt. Im Idealfall setzt man überhaupt auf Implementierungen offener Standards, weil denen des W3C-Konsortiums. Denn wenn man das tut, kann die konkrete technologische Umsetzung teilweise jederzeit ausgewechselt werden.