% Ausblick Fazit
\chapter{Ausblick und Fazit}
\label{sec:ausblick_fazit}

Im Zuge der Digitalisierung werden die Anforderungen an moderne \acp{App} immer höher. Zum einen muss die Anwendung benutzerfreundlich gestaltet sein, zum anderen immer den neuesten Entwicklungen folgen, die es im Bereich der Smartphones gibt. So muss man sich im Laufe der Entwicklung eines neuen Produktes oft entscheiden, ob man eine leichtgewichtige Anwendung schafft, die nur die Kernfunktionen übernimmt, für die sie ursprünglich gedacht wurde, oder ob die Entwicklung breiter ausfällt und man eine Anwendung schafft, die dem Nutzer mehr abnimmt, als nur die Kernfunktionen. Dabei muss am Anfang der Erarbeitung des Konzeptes immer der Grundsatz stehen, das Programm so aufzubauen, dass man im Entwicklungsprozess alle Möglichkeiten offen hat und genau das ist es, was das große Ziel dieser Arbeit ist. Eine Anwendung zu entwerfen, die anfangs die wichtigsten Grundlagen bereitstellt, die aber im Laufe der Zeit immer weiter entwickelt und erweitert werden kann.
\\
\linebreak
Der Fokus auf modulare und erweiterbare Anwendungen steht aktuell im Mittelpunkt der modernen Softwareentwicklung, deshalb sollte auch das das Ziel eines Projektes sein, das in einer Hochschule von Studierenden bearbeitet wird, die später mit bestem und modernstem Wissen in die Arbeitswelt einsteigen sollen. In den vorhergehenden Kapiteln dieser Arbeit wird mehrmals der Grundsatz der Erweiterbarkeit bei einer Microservice Architektur erwähnt und erläutert. Genau nach diesem Grundsatz wird die Hochschul-\ac{App} auch aufgebaut sein, das heißt, dass es in Zukunft die Möglichkeit geben wird, die Funktionen der Anwendung zu erweitern. Dieses Kapitel soll nun kurz beleuchten, welche weiteren Features noch möglich und sinnvoll sind.

\section{Ausblick\label{sec:ausblick}}

Einige der funktionalen Anforderungen aus den Kapiteln \ref{sec:anforderungen_ag}, \ref{sec:anforderungen_io} und \ref{sec:anforderungen_sz}, wurden in dieser Arbeit nicht weiter betrachtet. Wie bereits erwähnt, stellt die gewählte Architektur des Projektes die Möglichkeit dar, weitere Funktionen einzubinden, ohne etwas am bestehenden Programmcode ändern zu müssen. Die in dieser Bachelorarbeit nicht betrachteten möglichen Funktionen werden im folgenden aufgelistet.

\subsection*{Sprachumfang}

Im Rahmen der Internationalisierungsbemühungen der Hochschule Hof liegt es im Interesse des International Office, eine Hochschul-\ac{App} anbieten zu können, die in mehreren Sprachen erhältlich ist. Einige mögliche Vorschläge und favorisierte Sprachen der aktuellen Studierenden der Hochschule Hof wurden in Kapitel \ref{sec:anf_io} bereits aufgelistet. Zusätzlich zu den bereits im Funktionsumfang dieser Bachelorarbeit und der zugehörigen Praxisarbeit \footnote{Siehe Kapitel \ref{sec:nebenlektuere}} enthaltenen Sprachen Deutsch und Englisch kann die Anwendung ebenfalls um weitere Sprachen erweitert werden. Dazu könnte ein Konzept erstellt werden, das eine kontinuierliche Einbindung neuer Sprachen unterstützt. So könnten internationale Studierende schnell eigene Sprachen einpflegen. 

\subsection*{Verlinkung wichtiger Informationen}

Ein Wunsch, der aus den Umfragen in Kapitel \ref{sec:umfrage_erg} hervorgeht, ist die Verlinkung wichtiger Informationen zu den Vorlesungen, die im Stundenplan aus der Schnittstelle ausgelesen werden. Dazu gehören beispielsweise eine Referenz auf das entsprechende Modulhandbuch der Vorlesung oder auch die in Kapitel \ref{sec:anf_sz} erwähnten Anmeldungen zu den Sprachkursen. Diese Informationen sind in den vorhandenen Datenbanken der Hochschule nicht vorhanden, somit muss erst analysiert werden, wie man die passenden Informationen zu den Vorlesungen zuteilen kann.

\subsection*{Sprachkursinformationen}

Anders als nur die Verlinkung der Anmeldung zu den Sprachkursen verhält es sich mit den allgemeinen Informationen zu den Sprachkursen. Ein Sprachkurs an sich enthält weit mehr Informationen, als es normale Vorlesungen tun. Ein Sprachkurs hat vorausgesetzte Kenntnisse oder vorhergehende Sprachkurse, die erst abgelegt werden müssen, um teilnehmen zu dürfen. Zudem bieten viele Sprachkurse ebenfalls eine Zertifizierung an, welche am Ende des Kurses abgelegt werden kann. Zu den bereits genannten Punkten kommen noch weitere Informationen, die man über einen Sprachkurs wissen möchte. Diese anzuzeigen benötigt jedoch weitere Analysen, denn die Informationen werden in einer anderen Datenquelle gepflegt, als die regulären Vorlesungsinformationen. Weitere Anmerkungen zu diesem Problem wurden bereits im Kapitel \ref{sec:prob_sz} aufgeführt. 

\subsection*{Kursvorschläge}

Engagierte Studierende wünschen sich oft, sich auch außerhalb ihrer Pflichtveranstaltungen weiterbilden zu können. Im allgemeinen wurde in der Umfrage, die in Kapitel \ref{sec:umfrage} genauer betrachtet wurde, oft bemerkt, dass man sich wünscht, in freien Zeiten andere relevante Vorlesungen vorgeschlagen zu bekommen. Darunter fallen für die meisten Studierenden auch Sprachkurse. Die Einbindung einer solchen Funktion bedarf weiteren Entwicklungsaufwand, denn es wird ein Algorithmus benötigt, der mögliche Vorlesungen findet. Zudem wäre es ratsam, nicht jede Art von Vorlesung vorzuschlagen, sonder nur die, die tatsächlich relevant für den Nutzer ist. Dies bedarf einer Logik oder auch Intelligenz, die über den Rahmen dieser Arbeit hinaus geht.

\subsection*{Sondertermine}

Die Hochschule Hof bietet ein breites Spektrum an Veranstaltungen an, das weit über die regulären Vorlesungen der Studierenden hinausgeht. Oft werden Lesungen oder Seminare angeboten, die im eigentlichen Stundenplan nicht angezeigt werden. Außerdem halten oft auch externe Personen Vorträge über aktuelle Themen, auch diese werden lediglich auf der Website der Hochschule angezeigt oder per E-Mail bekannt geben. Ein mögliches Feature der Hochschul-Anwendung könnte es demnach sein, auch aktuelle Sondertermine mit anzuzeigen und dem Nutzer auch vorzuschlagen, falls es relevant für das Themengebiet seines Studiums ist.

\subsection*{Primuss-Einbindung}

Ein wichtiger Baustein im Alltag der Studierenden ist das Primuss-Portal der Hochschule Hof. Das Primuss-Portal ist ein Portal, bei dem die Studierenden alle organisatorischen Aufgaben für ihr Studium erledigen können. Dazu gehören beispielsweise Anträge und Formulare aus dem Studienbüro, das Ausdrucken von Studienbestätigungen, die Einrichtung der Bankverbindung für den Studienbeitrag und die Einsicht der vorläufigen Noten aus abgelegten Prüfungen. Eine Einbindung dieses Portals in einer Hochschul-\ac{App} würde den Nutzern der Anwendung viel Zeit dabei sparen, das Portal selbst aufzurufen und sich bei jedem Mal neu anmelden zu müssen. 

\subsection*{Kalendersynchronisation}

Wie bereits erwähnt wurde, hat die Hochschule Hof neben den eigentlichen Vorlesungen auch einige Sondertermine und auch andere Termine, die den Studienalltag betreffen, wie zum Beispiel den freien beweglichen Tag oder feste Termine, beispielsweise Semesterbeginn und Semesterende, sowie Rückmeldefristen und Prüfungszeiträume. Diese Termine, alternativ auch in Zusammenhang mit den eigentlichen Vorlesungen, können über eine Kalendersynchronisation in den Endgeräten und den privaten Konten der Nutzer automatisiert gespeichert werden. Dies bedarf einer Einbindung und Speicherung der Termine in einer Datenquelle und einen weiteren Service, über den die Daten abgefragt sowie synchronisiert werden können.
\\
\linebreak
Die Liste der Vorschläge ist nur ein Auszug aus den Möglichkeiten, die zur Erweiterung der Hochschul-\ac{App} umgesetzt werden könnten. Man erkennt bereits, dass dieses Projekt ein großes Potential für die Zukunft bietet, bei der die Hochschule Hof eine Anwendung vorzeigen kann, die den modernsten Ansprüchen und Techniken gerecht wird und die deren Studierenden eine einfache und mächtige Datenquelle für alles liefert, was sie für den Studienalltag benötigen.

\section{Weitere Bachelorarbeitsthemen}

Wie aus Kapitel \ref{sec:ausblick} hervorgeht, bietet das Resultat dieser Bachelorarbeit und des daraus entstehenden Projektes ein großes Potential an Erweiterungen. Anhand der Umfrage, die in Kapitel \ref{sec:umfrage} genauer betrachtet wurde, ist das Interesse der Studierenden an einer Hochschul-\ac{App}, die einen großen Funktionsumfang vorweisen kann, hoch. Eine stetige Verbesserung der Anwendung erhält das Interesse der Studierenden an der Hochschul-\ac{App} und erhält diese ebenfalls stets auf dem neuesten Stand der Entwicklungen, was die Langlebigkeit ebenfalls weit erhöht. Direkt am Anschluss dieser Arbeit lassen sich zwei neue Bachelorarbeitsthemen definieren, die die Qualität der \ac{App} auf lange Sicht steigern sollen.

\subsection*{Erweiterung der web-basierten Hochschul-\ac{App}}

Die bereits aufgelisteten Erweiterungen und möglichen weiteren Funktionen der Hoch\-schul-\ac{App} sind lediglich einige Vorschläge, die die Nützlichkeit der Anwendung stark verbessern würden. Jedoch verbleibt es nicht nur bei den genannten Erweiterungen, eine Analyse der möglichen weiteren Verbesserungen und die anschließende Entwicklung und Einbindung dieser würden die Hochschul-\ac{App} auf Dauer deutlich interessanter und attraktiver für die Studierenden der Hochschule Hof machen. Da die Grundfunktionen, für die die Anwendung eigentlich gedacht ist, bereits im Anschluss an diese Bachelorarbeit entwickelt werden, kann sich eine weitere Bachelorarbeit nur damit befassen, die Funktionen zu erweitern. Dabei sind die weiteren Funktionen auch nicht nur auf das oberflächliche Verhalten der \ac{App} beschränkt, sondern können sich auch auf andere Aspekte wie die Sicherheit, der Anwenderverwaltung und dem \ac{SSO} der Hochschule Hof beziehen. 

\subsection*{Verbesserung der Nutzerfreundlichkeit der Hochschul-\ac{App}}

Im Rahmen dieser Bachelorarbeit und der parallel dazu bearbeiteten Bachelorarbeit \footnote{Siehe Kapitel \ref{sec:nebenlektuere}} zum Thema Sicherheit, Nutzerverwaltung und der grafischen Oberfläche werden lediglich die Grundlagen für eine solide, langlebige und aktuelle Anwendung geschaffen. Die grundlegenden Funktionen werden implementiert und es wird eine erste Version der grafischen Benutzeroberfläche geschaffen. Jedoch liegt der Fokus bei der Entwicklung im Anfangsstadium der neuen web-basierten Hochschul-\ac{App} auf der Funktionalität und der Modularität. Es liegt also nahe, eine weitere Arbeit anzufertigen, die sich mit der Nutzerfreundlichkeit der Anwendung auseinander setzt. Dabei können Aspekte wie die Nutzbarkeit und die optische Wirkung der Oberfläche betrachtet werden, jedoch auch die Nutzung der Daten Schnittstelle, welche in Kapitel \ref{sec:appservices} definiert wurde, verbessert werden. Sinnvoll ist es hierbei, Nutzungsstatistiken zu sammeln, Befragungen der Studierenden durchzuführen und auf allgemeine Aspekte der \ac{UI} und \ac{UX} Entwicklung zu achten. Anhand solcher Verbesserungen kann eine Anwendung erschaffen werden, mit der die Studierenden nicht nur ihre Informationen einsehen können, sondern an der sie sich auch erfreuen können und die sie gern nutzen. Zudem wird somit eine Anwendung entwickelt, die als Aushängeschild für eine qualitativ hochwertige Entwicklung der Hochschule Hof dient.


\newpage
\section{Fazit\label{sec:fazit}}

Im Einklang mit der Digitalisierung, die auch im Alltag der Hochschulen immer präsenter wird, wurde in dieser Bachelorarbeit eine web-basierte Hochschul-\ac{App} entwickelt und erarbeitet, welche auch in weiterer Zukunft den Standards der Nutzer entsprechen wird. Dies wurde vor allem durch den modularen Aufbau und der Nutzung einiger Rahmenbedingungen, die die Langlebigkeit der \ac{App} deutlich unterstützen, erreicht. 
\\
\linebreak
Die Analyse der allgemeinen Zufriedenheit der Nutzer mit dem vorhandenen Anwendungsangebot der Hochschule Hof hat ergeben, dass die Akzeptanz der Nutzer für den aktuellen Anwendungen im allgemeinen hoch ist. Der Aufwand, der in die Entwicklung dieser Anwendungen geflossen ist, ist dadurch gerechtfertigt, dass über 90\% der Studierenden der Hochschule Hof eben einer dieser nativen, plattformgebundenen Anwendungen nutzen\footnote{Siehe Kapitel \ref{sec:umfrage_erg}}. Jedoch ging aus der selben Umfrage auch heraus, dass der Wunsch nach einer plattformübergreifenden \ac{App} genauso hoch ist, wie der Wunsch nach Erweiterungen und einer stabilen Anwendung, auf die sich die Studierenden beim Aufrufen wichtiger Informationen verlassen können.
\\
\linebreak
Ebenfalls hervor ging aus dieser Umfrage, dass die Internationalen Studierenden sich schwer tun, die \acp{App} vollständig zu nutzen. Darauf aufbauend gingen weitere funktionale Anforderungen aus den Bereichen des International Office und des Sprachenzentrums in die Liste der gewünschten Features der web-basierten Hochschul-\ac{App} ein. Zu diesen wurden ebenfalls die funktionalen Anforderungen des Auftraggebers mit aufgezählt. Durch die klare Linie des Auftraggebers bei der Sammlung der Anforderungen konnte eine klare Struktur für die Anwendung geschaffen werden. Anders als bei den funktionalen Anforderungen sind die Vorgaben des Auftraggebers in Sachen Umsetzung sehr offen, was eine perfekte Rahmenbedingung für die Analyse und die Entwicklung einer grundlegenden Architektur schaffte.
\\
\linebreak
Anhand des gegebenen Freiraums bei der Entwicklung konnte ein Konzept entwickelt werden, das den Anforderungen der Langlebigkeit und der Erweiterbarkeit in allen Ansprüchen gerecht wird. Mit der modularen Webservice Architektur wurde ein Konzept gefunden, das die funktionalen Anforderungen sowohl im Sourcecode, als auch im Deploymentprozess, klar trennt. Dies ermöglicht ein einfaches Austauschen der Module, auch Webservices genannt, falls dies nötig wird. Ebenfalls konnte so eine optimale Auslastung der Ressourcen gesichert werden, denn die einzelnen Webservices sind separat skalierbar und können auf verschiedenen logischen Adressen laufen. 
\\
\linebreak
Im Rahmen der zweiten zu diesem Thema angefertigten Bachelorarbeit konnte ebenfalls klar gemacht werden, wie einfach es ist, im Laufe der Jahre die optische Darstellung der Anwendung auszutauschen, da diese genau so modular aufgebaut ist, wie der Rest der Anwendung. Zudem können mehrere Anwendungen entwickelt werden, die nur auf die Daten der Hochschul-\ac{App} zugreifen. Dies wurde durch die Definition der \ac{REST}-Schnittstelle geschafft. Für die zukünftige, kontinuierliche und konsistente Weiterentwicklung der Anwendung, im speziellen der Schnittstelle, wurde ebenfalls ein Regelwerk geschaffen, welches das Design der Datenschnittstellen klar definiert. Anhand der bereits entwickelten Endpunkte dieser Schnittstelle können spätere Entwickler der \ac{App} klar erkennen, wie diese zu erweitern ist. 
\\
\linebreak
Wie aus Kapitel \ref{sec:ausblick} hervorgeht, wurde hier eine Grundlage einer Anwendung geschaffen, die in Zukunft deutlich weiter wachsen kann. Genau das sollte auch der Fokus der zuständigen Entwickler und des Auftraggebers sein, eine \ac{App} zu schaffen, die den Studierenden der Hochschule Hof den Alltag deutlich erleichtert. Konzentriert man sich dabei in Zukunft auf die in dieser Arbeit angebrachten Punkte und hält sich dabei an die beschriebenen Richtlinien, so steht einem erfolgreichen Projekt und der Erweiterung des Prototypen der web-basierten Hochschul-\ac{App} zu einer vollständigen und erfolgreichen Anwendung nichts im Wege.
